\def\weavesectionsss#1#2#3#4#5{\vfill\eject\noindent{\titlefont #2}\mark{#3}\vskip 0.5in\par\noindent\titlepage}
\def\weavesectionss#1#2#3#4#5{\vfill\eject\noindent{\stitlefont #2\hfill #5}\mark{#3}\bigskip\par\noindent\titlepage}
\def\weavesections#1#2#3#4#5{\bigskip\goodbreak\noindent{\bf #4#1. #2.}\mark{#3}\quad}
\def\weavesection#1#2#3#4#5{\bigskip\goodbreak\noindent{\bf #4#1.}\mark{#3}\quad}
\def\tweavesections#1#2#3#4#5{\bigskip\goodbreak\noindent{\bf #5.#4#1. #2.}\quad}
\def\tweavesection#1#2#3#4#5{\bigskip\goodbreak\noindent{\bf #5.#4#1.}\quad}
\def\nsweavesections#1#2#3#4#5{\noindent{\bf #4#1. #2.}\mark{#3}\quad}
\def\nsweavesection#1#2#3#4#5{\noindent{\bf #4#1.}\mark{#3}\quad}

\raggedbottom

\nopagenumbers
\newif\iftitle
\def\titlepage{\global\titletrue}
\headline={\iftitle\hfil\global\titlefalse\else\tenrm\hfil{\sourcefont \firstmark\qquad\the\pageno}\fi}

\titlepage

\def\quotesource#1{\medskip{\narrower\noindent{\sourcefont #1}\par}\smallskip}
\def\cast#1#2{{#1}$\;\rightarrow\;${#2}}
\def\comp#1#2{{#1}$\;\sim\;${#2}}
\def\al{$\alpha$}
\def\be{$\beta$}
\def\nonterminal#1{$\langle${\xreffont\pdfliteral direct{1 1 0 0 k}#1\special{PDF:0 g}}$\rangle$}
\def\fixed#1{{\bf #1}}
\def\from{\par\qquad\quad$\leftarrow$\quad}
\def\continuation{\par\qquad\quad\phantom{$\leftarrow$}\quad}

\font\tentex=cmtex10
\font\sourcefont=cmss10
\font\xreffont=cmss9
\hyphenchar\xreffont=-1
\font\usagefont=cmss8

\font\inchhigh=cminch
\font\sinchhigh=cminch at 60pt
\font\titlefont=cmss10 at 24pt
\font\stitlefont=cmss10 at 20pt
\font\sstitlefont=cmss10 at 20pt
\font\ssstitlefont=cmr10 at 12pt

\font\ninerm=cmr9
\font\eightrm=cmr8
\font\sixrm=cmr6

\font\ninei=cmmi9
\font\eighti=cmmi8
\font\sixi=cmmi6
\skewchar\ninei='177 \skewchar\eighti='177 \skewchar\sixi='177

\font\ninesy=cmsy9
\font\eightsy=cmsy8
\font\sixsy=cmsy6
\skewchar\ninesy='60 \skewchar\eightsy='60 \skewchar\sixsy='60

\font\sevenss=cmss10 at 7pt
\font\eightss=cmssq8

\font\eightssi=cmssqi8

\font\ninebf=cmbx9
\font\eightbf=cmbx8
\font\sixbf=cmbx6

\font\ninett=cmtt9
\font\eighttt=cmtt8

\hyphenchar\tentt=-1 % inhibit hyphenation in typewriter type
\hyphenchar\ninett=-1
\hyphenchar\eighttt=-1

\font\ninesl=cmsl9
\font\eightsl=cmsl8

\font\nineit=cmti9
\font\eightit=cmti8

\font\tenu=cmu10 % unslanted text italic
\font\magnifiedfiverm=cmr5 at 10pt
\font\manual=manfnt % font used for the METAFONT logo, etc.
\font\cmman=cmman % font used for miscellaneous Computer Modern variations

\newskip\ttglue
\def\tenpoint{\def\rm{\fam0\tenrm}%
  \textfont0=\tenrm \scriptfont0=\sevenrm \scriptscriptfont0=\fiverm
  \textfont1=\teni \scriptfont1=\seveni \scriptscriptfont1=\fivei
  \textfont2=\tensy \scriptfont2=\sevensy \scriptscriptfont2=\fivesy
  \textfont3=\tenex \scriptfont3=\tenex \scriptscriptfont3=\tenex
  \def\it{\fam\itfam\tenit}%
  \textfont\itfam=\tenit
  \def\sl{\fam\slfam\tensl}%
  \textfont\slfam=\tensl
  \def\bf{\fam\bffam\tenbf}%
  \textfont\bffam=\tenbf \scriptfont\bffam=\sevenbf
   \scriptscriptfont\bffam=\fivebf
  \def\tt{\fam\ttfam\tentt}%
  \textfont\ttfam=\tentt
  \tt \ttglue=.5em plus.25em minus.15em
  \normalbaselineskip=12pt
  \def\MF{{\manual META}\-{\manual FONT}}%
  \let\sc=\eightrm
  \let\big=\tenbig
  \setbox\strutbox=\hbox{\vrule height8.5pt depth3.5pt width\z@}%
  \normalbaselines\rm}

\def\ninepoint{\def\rm{\fam0\ninerm}%
  \textfont0=\ninerm \scriptfont0=\sixrm \scriptscriptfont0=\fiverm
  \textfont1=\ninei \scriptfont1=\sixi \scriptscriptfont1=\fivei
  \textfont2=\ninesy \scriptfont2=\sixsy \scriptscriptfont2=\fivesy
  \textfont3=\tenex \scriptfont3=\tenex \scriptscriptfont3=\tenex
  \def\it{\fam\itfam\nineit}%
  \textfont\itfam=\nineit
  \def\sl{\fam\slfam\ninesl}%
  \textfont\slfam=\ninesl
  \def\bf{\fam\bffam\ninebf}%
  \textfont\bffam=\ninebf \scriptfont\bffam=\sixbf
   \scriptscriptfont\bffam=\fivebf
  \def\tt{\fam\ttfam\ninett}%
  \textfont\ttfam=\ninett
  \tt \ttglue=.5em plus.25em minus.15em
  \normalbaselineskip=11pt
  \def\MF{{\manual hijk}\-{\manual lmnj}}%
  \let\sc=\sevenrm
  \let\big=\ninebig
  \setbox\strutbox=\hbox{\vrule height8pt depth3pt width\z@}%
  \normalbaselines\rm}

\def\ttninepoint{\def\rm{\fam0\ninerm}%
  \textfont0=\ninerm \scriptfont0=\sixrm \scriptscriptfont0=\fiverm
  \textfont1=\ninei \scriptfont1=\sixi \scriptscriptfont1=\fivei
  \textfont2=\ninesy \scriptfont2=\sixsy \scriptscriptfont2=\fivesy
  \textfont3=\tenex \scriptfont3=\tenex \scriptscriptfont3=\tenex
  \def\it{\fam\itfam\nineit}%
  \textfont\itfam=\nineit
  \def\sl{\fam\slfam\ninesl}%
  \textfont\slfam=\ninesl
  \def\bf{\fam\bffam\ninebf}%
  \textfont\bffam=\ninebf \scriptfont\bffam=\sixbf
   \scriptscriptfont\bffam=\fivebf
  \def\tt{\fam\ttfam\ninett}%
  \textfont\ttfam=\ninett
  \normalbaselineskip=11pt
  \normalbaselines\rm}

\def\eightpoint{\def\rm{\fam0\eightrm}%
  \textfont0=\eightrm \scriptfont0=\sixrm \scriptscriptfont0=\fiverm
  \textfont1=\eighti \scriptfont1=\sixi \scriptscriptfont1=\fivei
  \textfont2=\eightsy \scriptfont2=\sixsy \scriptscriptfont2=\fivesy
  \textfont3=\tenex \scriptfont3=\tenex \scriptscriptfont3=\tenex
  \def\it{\fam\itfam\eightit}%
  \textfont\itfam=\eightit
  \def\sl{\fam\slfam\eightsl}%
  \textfont\slfam=\eightsl
  \def\bf{\fam\bffam\eightbf}%
  \textfont\bffam=\eightbf \scriptfont\bffam=\sixbf
   \scriptscriptfont\bffam=\fivebf
  \def\tt{\fam\ttfam\eighttt}%
  \textfont\ttfam=\eighttt
  \tt \ttglue=.5em plus.25em minus.15em
  \normalbaselineskip=9pt
  \def\MF{{\manual opqr}\-{\manual stuq}}%
  \let\sc=\sixrm
  \let\big=\eightbig
  \setbox\strutbox=\hbox{\vrule height7pt depth2pt width\z@}%
  \normalbaselines\rm}

\def\tenmath{\tenpoint\fam-1 } % use after $ in ninepoint sections
\def\tenbig#1{{\hbox{$\left#1\vbox to8.5pt{}\right.\n@space$}}}
\def\ninebig#1{{\hbox{$\textfont0=\tenrm\textfont2=\tensy
  \left#1\vbox to7.25pt{}\right.\n@space$}}}
\def\eightbig#1{{\hbox{$\textfont0=\ninerm\textfont2=\ninesy
  \left#1\vbox to6.5pt{}\right.\n@space$}}}

\catcode`@=11 % borrow the private macros of PLAIN (with care)

\def\|{\leavevmode\hbox{\tt\char`\|}} % vertical line

% macros for verbatim scanning
\chardef\other=12
\def\ttverbatim{\begingroup\ttninepoint\frenchspacing
  \catcode`\\=\other
  \catcode`\{=\other
  \catcode`\}=\other
  \catcode`\$=\other
  \catcode`\&=\other
  \catcode`\#=\other
  \catcode`\%=\other
  \catcode`\~=\other
  \catcode`\_=\other
  \catcode`\^=\other
  \obeyspaces \obeylines \tt}

\outer\def\begintt{$$\let\par=\endgraf \ttverbatim \parskip=\z@
  \catcode`\|=0 \rightskip-5pc \ttfinish}
{\catcode`\|=0 |catcode`|\=\other % | is temporary escape character
  |obeylines % end of line is active
  |gdef|ttfinish#1^^M#2\endtt{#1|vbox{#2}|endgroup|tenpoint$$}}

\catcode`\|=\active
{\obeylines \gdef|{\ttverbatim \spaceskip\ttglue \let^^M=\  \let|=\endgroup}}

% macros for syntax rules (again, not in Appendix E)
\def\[#1]{\silenttrue\xref|#1|\thinspace{\tt#1}\thinspace} % keyword in syntax
\def\beginsyntax{\endgraf\nobreak\medskip
  \begingroup \catcode`<=13 \catcode`[=13
  \let\par=\endsyntaxline \obeylines}
\def\endsyntaxline{\futurelet\next\syntaxswitch}
\def\syntaxswitch{\ifx\next\<\let\next=\syntaxrule
  \else\ifx\next\endsyntax\let\next=\endgroup
  \else\let\next=\continuerule\fi\fi \next}
\def\continuerule{\hfil\break\indent\qquad}
\def\endsyntax{\medbreak\noindent}
{\catcode`<=13 \catcode`[=13
  \global\let<=\< \global\let[=\[
  \gdef\syntaxrule<#1>{\endgraf\indent\silentfalse\xref\<#1>}}
\def\is{\ $\longrightarrow$ }
\def\alt{\ $\vert$ }

% macros to demarcate lines quoted from TeX source files
\def\beginlines{\par\begingroup\nobreak\medskip\parindent\z@ \obeylines
\nobreak \everypar{\strut}}
\def\endlines{\endgroup\medbreak\noindent}


\phantom{A}
\vskip 0.5in
\noindent{{\titlefont The Goldbach Conjecture}}
\vskip 0.5in

\noindent{{\stitlefont Sections}}
\vskip 0.1in
\par\noindent {\it }
\vskip 0.2in
\noindent{{\sstitlefont Summing Primes}}
\vskip 0.1in
\par\noindent {\it Here we verify the conjecture for small numbers.}

\vskip 0.2in
\noindent{{\sstitlefont The Sieve of Eratosthenes}}
\vskip 0.1in
\par\noindent {\it A fairly fast way to determine if small numbers are prime, given storage.}


% Weave of 'Complete Program' generated by Inweb
\weavesectionss{}{Summing Primes}{Summing Primes\quad$\P$}{\P}{}%
\smallskip\par\noindent{\it Here we verify the conjecture for small numbers.}\smallskip\noindent
\weavesection{1}{Dunno}{\quad$\S$1}{\S}{S/sp}%
On 7 June 1742, Christian Goldbach wrote a letter from Moscow to Leonhard
Euler in Berlin making ``eine conjecture hazardiren'' that every even number
greater than $2$ can be written as a sum of two primes.[1] Euler did not
know if this was true, and nor does anyone else.
Goldbach, a professor at St Petersburg and tutor to Tsar Peter II, wrote in
several languages in an elegant cursive script, and was much valued as a
letter-writer, though his reputation stands less high today.[2] All the same,
the general belief now is that primes are just plentiful enough, and just
evenly-enough spread, for Goldbach to be right. It is known that:
\item{(a)}every even number is a sum of at most six primes (Ramaré, 1995), and
\item{(b)}every odd number is a sum of at most five (Tao, 2012).


[1] ``Greater than 2'' is our later proviso: Goldbach needed no such exception
because he considered 1 a prime number, as was normal then, and was sometimes
said as late as the early twentieth century.

[2] Goldbach, almost exactly a contemporary of Voltaire, was a good citizen
of the great age of Enlightenment letter-writing. He and Euler exchanged
scholarly letters for over thirty years, not something Euler would have
kept up with a duffer. Goldbach was also not, as is sometimes said, a lawyer.
See: http://mathshistory.st-andrews.ac.uk/Biographies/Goldbach.html
An edited transcription of the letter is at: http://eulerarchive.maa.org//correspondence/letters/OO0765.pdf
\weavesection{2}{Dunno}{\quad$\S$2}{\S}{S/sp}%
Computer verification has been made up to around $10^{18}$, but by rather
better methods than the one we use here. We will only go up to:

\beginlines
\smallskip\par\noindent ||{\ninebf define} |RANGE|\special{PDF:0 g}| 100|\special{PDF:0 g}||
\endlines
\beginlines
\smallskip\par\noindent |#include|\special{PDF:0 g}| <stdio|\special{PDF:0 g}|.h|\special{PDF:0 g}|>|
\smallskip
\smallskip\par\noindent |int|\special{PDF:0 g}| |\pdfliteral direct{0 1 1 0 k}|main|\special{PDF:0 g}|(int|\special{PDF:0 g}| argc|\special{PDF:0 g}|, char|\special{PDF:0 g}| *argv|\special{PDF:0 g}|[]) {|
\smallskip\par\noindent |    for|\special{PDF:0 g}| (int|\special{PDF:0 g}| i|\special{PDF:0 g}|=4; i|\special{PDF:0 g}|<RANGE|\special{PDF:0 g}|; i|\special{PDF:0 g}|=i|\special{PDF:0 g}|+2)  |\hfill{\ttninepoint\it  stepping in twos to stay even}||
\smallskip\par\noindent |        |\pdfstartlink attr{/C [0.9 0 0] /Border [0 0 0]} goto num 100 $\langle${\xreffont\pdfliteral direct{1 1 0 0 k}Solve Goldbach's conjecture for i {\sevenss 2.1}}\special{PDF:0 g}$\rangle$ \pdfendlink|;|\special{PDF:0 g}||
\smallskip\par\noindent |}|\special{PDF:0 g}||
\endlines
\weavesection{2.1}{Dunno}{\quad$\S$2.1}{\S}{S/sp}%
This ought to print:

\beginlines
\smallskip\par\noindent |    $|\special{PDF:0 g}| |\pdfliteral direct{0 1 1 0 k}|goldbach/Tangled/goldbach|\special{PDF:0 g}||
\smallskip\par\noindent |    4 = 2+2|
\smallskip\par\noindent |    6 = 3+3|
\smallskip\par\noindent |    8 = 3+5|
\smallskip\par\noindent |    10 = 3+7 = 5+5|
\smallskip\par\noindent |    12 = 5+7|
\smallskip\par\noindent |    14 = 3+11 = 7+7|
\smallskip\par\noindent |    ...|
\endlines
We'll print each different pair of primes adding up to $i$. We
only check in the range $2 \leq j \leq i/2$ to avoid counting pairs
twice over (thus $8 = 3+5 = 5+3$, but that's hardly two different ways).

\smallskip\par\noindent ||\pdfdest num 100 fit $\langle${\xreffont\pdfliteral direct{1 1 0 0 k}Solve Goldbach's conjecture for i {\sevenss 2.1}}\special{PDF:0 g}$\rangle$ $\equiv$||

\beginlines
\smallskip\par\noindent |    printf|\special{PDF:0 g}|("%d"|\special{PDF:0 g}|, i|\special{PDF:0 g}|);|
\smallskip\par\noindent |    for|\special{PDF:0 g}| (int|\special{PDF:0 g}| j|\special{PDF:0 g}|=2; j|\special{PDF:0 g}|<=i|\special{PDF:0 g}|/2; j|\special{PDF:0 g}|++)|
\smallskip\par\noindent |        if|\special{PDF:0 g}| ((isprime(j|\special{PDF:0 g}|)) && (isprime(i|\special{PDF:0 g}|-j|\special{PDF:0 g}|)))|
\smallskip\par\noindent |            printf|\special{PDF:0 g}|(" = %d+%d"|\special{PDF:0 g}|, j|\special{PDF:0 g}|, i|\special{PDF:0 g}|-j|\special{PDF:0 g}|);|
\smallskip\par\noindent |    printf|\special{PDF:0 g}|("\n"|\special{PDF:0 g}|);|
\endlines
\par\noindent\penalty10000
{\usagefont This code is used in $\S$2.}\smallskip
\weavesectionss{}{The Sieve of Eratosthenes}{The Sieve of Eratosthenes\quad$\P$}{\P}{}%
\smallskip\par\noindent{\it A fairly fast way to determine if small numbers are prime, given storage.}\smallskip\noindent
\medskip\hrule\smallskip\par\noindent{\usagefont S1~Storage; S2~Primality}\par\medskip\hrule\bigskip
\weavesection{1}{Dunno}{\quad$\S$1}{\S}{S/tsoe}%
This technique, still essentially the best sieve for finding prime
numbers, is attributed to Eratosthenes of Cyrene and dates from the 200s BC.
Since composite numbers are exactly those numbers which are multiples of
something, the idea is to remove everything which is a multiple: whatever
is left, must be prime.
\smallskip\par\noindent%
This is very fast (and can be done more quickly than the implementation
below), but (a) uses storage to hold the sieve, and (b) has to start right
back at 2 - so it can't efficiently test just, say, the eight-digit numbers
for primality.

\beginlines
\smallskip\par\noindent |int|\special{PDF:0 g}| still_in_sieve|\special{PDF:0 g}|[RANGE|\special{PDF:0 g}| + 1|\special{PDF:0 g}|];|
\smallskip\par\noindent |int|\special{PDF:0 g}| sieve_performed|\special{PDF:0 g}| = FALSE|\special{PDF:0 g}|;|
\endlines
\weavesection{2}{Dunno}{\quad$\S$2}{\S}{S/tsoe}%
We provide this as a function which determines whether a number is prime:

\beginlines
\smallskip\par\noindent ||{\ninebf define} |TRUE|\special{PDF:0 g}| 1|\special{PDF:0 g}||
\smallskip\par\noindent ||{\ninebf define} |FALSE|\special{PDF:0 g}| 0|\special{PDF:0 g}||
\endlines
\beginlines
\smallskip\par\noindent |int|\special{PDF:0 g}| |\pdfliteral direct{0 1 1 0 k}|isprime|\special{PDF:0 g}|(int|\special{PDF:0 g}| n|\special{PDF:0 g}|) {|
\smallskip\par\noindent |    if|\special{PDF:0 g}| (n|\special{PDF:0 g}| <= 1|\special{PDF:0 g}|) return|\special{PDF:0 g}| FALSE|\special{PDF:0 g}|;|
\smallskip\par\noindent |    if|\special{PDF:0 g}| (n|\special{PDF:0 g}| > RANGE|\special{PDF:0 g}|) { printf|\special{PDF:0 g}|("Out of range!\n"|\special{PDF:0 g}|); return|\special{PDF:0 g}| FALSE|\special{PDF:0 g}|; }|
\smallskip\par\noindent |    if|\special{PDF:0 g}| (!sieve_performed|\special{PDF:0 g}|) |\pdfstartlink attr{/C [0.9 0 0] /Border [0 0 0]} goto num 101 $\langle${\xreffont\pdfliteral direct{1 1 0 0 k}Perform the sieve {\sevenss 2.1}}\special{PDF:0 g}$\rangle$ \pdfendlink|;|
\smallskip\par\noindent |    return|\special{PDF:0 g}| still_in_sieve|\special{PDF:0 g}|[n|\special{PDF:0 g}|];|
\smallskip\par\noindent |}|
\endlines
\weavesection{2.1}{Dunno}{\quad$\S$2.1}{\S}{S/tsoe}%
We save a little time by noting that if a number up to ||\special{PDF:0 g}|RANGE|\special{PDF:0 g}|| is composite
then one of its factors must be smaller than the square root of ||\special{PDF:0 g}|RANGE|\special{PDF:0 g}||. Thus,
in a sieve of size 10000, one only needs to remove multiples of 2 up to 100,
for example.

\smallskip\par\noindent ||\pdfdest num 101 fit $\langle${\xreffont\pdfliteral direct{1 1 0 0 k}Perform the sieve {\sevenss 2.1}}\special{PDF:0 g}$\rangle$ $\equiv$||

\beginlines
\smallskip\par\noindent |    |\pdfstartlink attr{/C [0.9 0 0] /Border [0 0 0]} goto num 102 $\langle${\xreffont\pdfliteral direct{1 1 0 0 k}Start with all numbers from 2 upwards in the sieve {\sevenss 2.1.1}}\special{PDF:0 g}$\rangle$ \pdfendlink|;|
\smallskip\par\noindent |    for|\special{PDF:0 g}| (int|\special{PDF:0 g}| n|\special{PDF:0 g}|=2; n|\special{PDF:0 g}|*n|\special{PDF:0 g}| <= RANGE|\special{PDF:0 g}|; n|\special{PDF:0 g}|++)|
\smallskip\par\noindent |        if|\special{PDF:0 g}| (still_in_sieve|\special{PDF:0 g}|[n|\special{PDF:0 g}|])|
\smallskip\par\noindent |            |\pdfstartlink attr{/C [0.9 0 0] /Border [0 0 0]} goto num 103 $\langle${\xreffont\pdfliteral direct{1 1 0 0 k}Shake out multiples of n {\sevenss 2.1.2}}\special{PDF:0 g}$\rangle$ \pdfendlink|;|
\smallskip\par\noindent |    sieve_performed|\special{PDF:0 g}| = TRUE|\special{PDF:0 g}|;|
\endlines
\par\noindent\penalty10000
{\usagefont This code is used in $\S$2.}\smallskip
\weavesection{2.1.1}{Dunno}{\quad$\S$2.1.1}{\S}{S/tsoe}%
\smallskip\par\noindent ||\pdfdest num 102 fit $\langle${\xreffont\pdfliteral direct{1 1 0 0 k}Start with all numbers from 2 upwards in the sieve {\sevenss 2.1.1}}\special{PDF:0 g}$\rangle$ $\equiv$||

\beginlines
\smallskip\par\noindent |    still_in_sieve|\special{PDF:0 g}|[1] = FALSE|\special{PDF:0 g}|;|
\smallskip\par\noindent |    for|\special{PDF:0 g}| (int|\special{PDF:0 g}| n|\special{PDF:0 g}|=2; n|\special{PDF:0 g}| <= RANGE|\special{PDF:0 g}|; n|\special{PDF:0 g}|++) still_in_sieve|\special{PDF:0 g}|[n|\special{PDF:0 g}|] = TRUE|\special{PDF:0 g}|;|
\endlines
\par\noindent\penalty10000
{\usagefont This code is used in $\S$2.1.}\smallskip
\weavesection{2.1.2}{Dunno}{\quad$\S$2.1.2}{\S}{S/tsoe}%
\smallskip\par\noindent ||\pdfdest num 103 fit $\langle${\xreffont\pdfliteral direct{1 1 0 0 k}Shake out multiples of n {\sevenss 2.1.2}}\special{PDF:0 g}$\rangle$ $\equiv$||

\beginlines
\smallskip\par\noindent |    for|\special{PDF:0 g}| (int|\special{PDF:0 g}| m|\special{PDF:0 g}|= n|\special{PDF:0 g}|+n|\special{PDF:0 g}|; m|\special{PDF:0 g}| <= RANGE|\special{PDF:0 g}|; m|\special{PDF:0 g}| += n|\special{PDF:0 g}|) still_in_sieve|\special{PDF:0 g}|[m|\special{PDF:0 g}|] = FALSE|\special{PDF:0 g}|;|
\endlines
\par\noindent\penalty10000
{\usagefont This code is used in $\S$2.1.}\smallskip
% End of weave
\end

